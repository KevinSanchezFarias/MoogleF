\documentclass[10pt]{extarticle}
\title{Informe escrito Proyecto de Programación 1er. Año Ciencias de la Computación }
\date{19-07-2023}
\author{Jose Ernesto Morales Lazo C-122}

\usepackage[spanish]{babel}
\usepackage[utf8]{inputenc}
\usepackage{graphicx}
\usepackage{xcolor}
\usepackage{multicol}
\usepackage{hyperref}
\bibliographystyle{plain}


\begin{document}
	\title{Informe escrito Proyecto de Programación 1er. Año Ciencias de la Computación }
	\author{Jose Ernesto Morales Lazo C-122}
	\date{19-07-20	}


	
	\maketitle
	\vspace{5cm}
	\tableofcontents
	
	
	\newpage
	\begin{abstract}
		El contenido de este informe corresponde a la realizacion del primer proyecto de Programacion de Primer Año de Ciencias de la Computacion . A continuacion se proveeran al lector cuestiones fundamentales en pos de comprender el proyecto como son las clases que componen el codigo del programa junto  con su funcion o proposito en el mismo . \textcolor{yellow}{\rule{10.50cm}{2pt}}
		
	\end{abstract}
	
	
	\section{Introducción}\label{sec:introduccion}
    \begin{multicols}{2}
    	En este proyecto se presenta la realización de un navegador llamado Moogle . Su propósito es ofrecer al usuario la posibilidad de realizar una busqueda sobre un tema de interes y recibir los textos que más posibilidades tengan de satisfacer el contenido de su busqueda . Es un proyecto muy provechoso ya que se trata de algún modo emular una herramienta tan presente en el dia a dia de las personas de nuestro tiempo como son los navegadores . Su presición y rapidez la hacen la enciclopedia por excelencia de nuestro tiempo y un muy buen ayudante en el momento de ayudar a resolver nuestras dudas sobre casi cualquier tema . El proyecto trata , en esencia , de implentar una tecnica de procesamiento de lenguaje natural como lo es el TD-IDF . Este se encarga de evaluar la importancia de una palabra en un documento o una colección de documentos . 
    \end{multicols}
    
    \section{Clases}\label{sec:clases} 
    
    El namespace Moogle Engine cuenta con 6 clases. La clase Documentos, la clase Puntuacion , la 
    clase Matriz y la clase Query – estas cuatro implementadas por mi – y las clases SearchItem y 
    SearchResult.
    
    \subsection{Clase \texttt{documentos}}\label{sub:documentos}
    Esta clase es la que se encarga de procesar cada Documento de la colección, un objeto tipo 
    Documento es en esencia una lista de listas donde cada lista corresponde a un documento de la 
    colección . Cada lista es el resultado del procesamiento de cada texto, dígase eliminar caracteres 
    especiales, espacios y demás cuestiones que no favorecerían al desarrollo del programa, es así que 
    dicha lista contiene las palabras de cada texto .
    
   \subsubsection{Métodos}
   
   \begin{enumerate}
   	\item  Constructor : se inicializa un objeto tipo texto, lo hace primero inicializando una lista 
   	de listas de string a la cual  se le aplica el método RecorrerDocumento, el mismo se explica detalladamente a continuación. 
   	\item RecorrerDocumento : Este método devuelve una lista de listas de string y es el encargado de procesar los textos y 
   	dividirlos en palabras , de manera que se eliminen los caracteres especiales, espacios, etc. 
   	Se utiliza el método GetFiles de la clase Directory para extraer las rutas de todos los archivos de la 
   	carpeta Content y archivarlas en un array de string. En un ciclo for se recorre este array , en este 
   	ciclo primero se inicializa una lista que a la postre se va a añadir a la lista de listas que se va a 
   	devolver . En segundo lugar se emplea el método ReadAllText de la clase File para almacenar en 
   	una cadena cada texto y en tercero se inicializa un array en el que usando la clase Regex y una 
   	expresión regular de la misma que define los delimitadores que se toman como referencia para 
   	hacer Split en la cadena en la que anteriormente se almaceno el texto. Ya por último se recorre 
   	este array , en primer lugar se aplica el método ToLower a cada elemento y y el método 
   	IsNullOrEmpty , con el objetivo de que no pasen cadenas vacias (que serian los espacios que 
   	contenia el texto) y además que las palabras pasen en minúsculas a la lista a las que se añaden en 
   	este ciclo . Dicha lista se añade a la lista de listas y ya tendríamos los textos procesados palabra 
   	por palabra y listos para utilizarlos .
   	 
   	\item  Repeticiones :  Es un método que devuelve un diccionario con 
   	clave string y valor  double . En un método de instancia que se le aplica a un objeto tipo 
   	Documentos con la finalidad de almacenar cuantas veces se repite una palabra en un texto 
   	determinado. Para esto se recorre cada lista de la lista de listas de string (Objeto Tipo 
   	Documentos) y  empleando condicionales se hace lo siguiente: Si la palabra ya esta en el 
   	diccionario se aumenta en uno su valor en el mismo , si no se añade al mismo con valor uno . De 
   	esta manera cada vez que aparezca la palabra en el texto su valor en el diccionario va a aumentar. 
   	Cuando este ciclo se acaba se añade el diccionario a la lista de diccionarios. Al finalizar tendremos 
   	una lista con un diccionario por cada texto, en el que aparece la palabra del texto y cuantas veces 
   	se repite en él. Esto no tiene ningún interés a la hora de procesar los textos, pero si lo tendrá a la 
   	hora de hacer los cálculos del Term Frequency en la clase Matriz. Lo mismo pasa con el método 
   	Vocabulario que se describe a continuación. 
    \item Vocabulario :	Recorre un objeto tipo texto y almacena en un Hash Set de string todas las palabras 
   	de cada lista sin que se repiten. Se emplea el HashSet por su utilidad para almacenar elementos 
   	que no se desea que se repitan. Esto será de utilidad en los cálculos posteriores de la clase Matriz. 
   	\item  Index :  devuelve el Count de una lista de un objeto tipo 
   	Documentos -Lista de listas - .Se le pasa por parámetro la posición de la lista de la cual se desea 
   	conocer la posición . 
   	\item  CopiarDocumentos :  Realiza una copia de un objeto tipo 
   	Documentos, esto para evitar problemas que se pueden dar a la hora de recorrer estos objetos en 
   	ciclo.
   \end{enumerate}
   
   
    \subsection{Clase \texttt{matriz}}\label{sub:matriz}
    
    La clase Matriz cuenta con un método constructor en el cual se inicializa un objeto tipo 
    Documentos que se le pasará como parámetro al método CalcularTDIDF el cual da valor a la 
    variable tfidf – una lista de diccionarios con clave string y valor double -, la cual será nuestro 
    objeto tipo Matriz propiamente dicho. Este objeto contendrá los valores del TFIDF de cada palabra 
    en cada texto.
    
    \subsubsection{Métodos}
   
   \begin{enumerate}
    
   \item  TFDoc: Lo primero que se hace en este método es llamar al método Repeticiones de la 
   clase Documentos explicado anteriormente, así tendremos la frecuencia de 
   cada palabra de cada texto. Luego recorremos la lista de diccionarios procedente del método 
   Repeticiones. En otro bucle recorremos cada diccionario y calculamos el tf de cada clave: el valor 
   de la misma (su frecuencia en el texto dividido) dividido por el texto. Index – devuelve el Count de 
   la lista en cuestión, es decir, la cantidad de palabras del texto y se le pasa por parámetro un 
   contador el cual representa la posición de la lista de cuyas palabras se le calcula el TF -. Cada 
   resultado se le añade a un diccionario que se inicializa en el primer bucle. Dicho diccionario al final 
   el proceso se añade a la lista de diccionarios. Es así como se obtiene la lista de diccionarios, con un 
   diccionario por texto y con valor del TF de sus palabras. 
   \item Copiar Hash: de esta clase recibe un Hashset como parámetro, el cual será el que 
   contenga todas las palabras del vocabulario. Devuelve un diccionario con clave string (todas las 
   palabras del vocabulario) y valor 0, es cero porque es un diccionario cuya utilidad se basa en 
   aumentar su frecuencia en el método siguiente. 
   \item NumeroenDocumentos : es el método empleado para saber en cuantos de los textos de 
   la colección se repite cada palabra del vocabulario. Para esto se utiliza un HashSet con todas las 
   palabras del vocabulario (Método Vocabulario de la clase Documentos) y un diccionario con todas 
   estas palabras y valor 0 (Método Copiar Hash). Usando el método CopiarDocumentos se realiza 
   una copia del objeto tipo Documentos ( Lista de listas con todas las palabra de los textos ) , esto 
   para poder recorrer dicho objeto .Luego se itera sobre esta copia y en cada ciclo se crea un 
   HashSet (llamémoslo palabrasTexto) que será el que contendrá todas las palabras de cada texto ( 
   las palabras de cada lista ). Luego en un bucle dentro de este se itera sobre el HashSet procedente 
   del método Vocabulario ( todas las palabras de la colección ) y empleando el método Contains al 
   HashSet palabrasTexto y si devuelve true aumentamos en 1 el valor de la palabra en el diccionario 
   frecuencias ( procedente de la llamada al método CopiarHash ) . De esta manera se obtiene en 
   cuantos documentos aparece cada palabra de la colección de textos .
   \item IDF : lo que se hace es después de llamar al anterior método usar estos datos para 
   calcular el IDF con formula logaritmo natural del cociente de la cantidad total de documentos en la 
   colección entre la cantidad de documentos en los que aparece la palabra en cuestión. En mi caso 
   yo lo calcule como resta de logaritmos de igual base ya que la cantidad de documentos va a ser 
   constante ya le calculo el valor afuera del ciclo, a dicho valor se le resta el logaritmo natural de la 
   cantidad de documentos en los que 
   aparece la palabra. Luego este valor se almacena en un diccionario en el cual estarán todos los 
   valores del IDF del vocabulario. 
   \item TDIDFDoc : Devuelve el objeto tipo Matriz propiamente dicho. En  un ciclo se recorre la lista de diccionarios que almacena los valores del TF de cada palabra de cada documento, en otro ciclo más interno se recorre cada diccionario y se multiplica el valor de cada 
   clave (el TF de cada palabra) por el valor del IDF de la palabra en el diccionario que almacena los mismos. Estos valores se almacenan en un diccionario que posteriormente se añaden a la lista de  diccionarios. De esta manera se obtiene una lista de diccionarios, un diccionario para cada texto en el cual se almacena el TF-IDF de cada palabra. 
   
\end{enumerate}

\subsection{Clase \texttt{Query}}\label{sub:query}

La utilidad de la implementación de esta clase el proyecto se basa en el procesamiento de la 
query. Aunque su manejo puede ser muy similar al de los de los documentos considere oportuno 
procesarla en una clase aparte.

\subsubsection{Métodos}

\begin{enumerate}
	\item Método Constructor : El objeto tipo Query consta de un constructor el cual recibe como parámetro la query insertada 
	por el usuario . Cuando se llama al método constructor se le da valor a un Diccionario con el 
	método CalcularTFIDF (recibe como parámetro la 	query del usuario que se la pasa como parámetro al constructor. 
	\item  CopiarQuery :Realiza una copia de un objeto tipo Query para 	evitar errores a la hora de recorrerlos mediante un ciclo en otros métodos. 
	\item CogerPalabras : Se utiliza para procesar la query  . Este recibe como parámetro la query 
	del usuario y usando una expresión regular de la clase Regex -que determina los separadores que 
	se deben tomar como referencias para hacer Split ( dígase comas , comillas , paréntesis y demás ) 
	– se almacena en un array de string cada palabra por separado y ya sin los caracteres especiales . 
	Luego se recorre este array y aplicando el método IsNullOrEmpty (para evitar que los espacios 
	pasen a la lista con las palabras de la query que va a devolver este método) se agrega el elemento 
	a la lista .Se agrega en minúsculas después de aplicar el método ToLower a la palabra . Es así que a 
	partir de este método se obtiene una lista con las palabras de la query . 
	\item  Repeticiones : Este se usa para ver cuántas veces se repite cada palabra de la 
	query en dicha query. Se recibe como parámetro la lista de las palabras de la query y se itera sobre 
	ella, con el método ContainsKey se comprueba si la palabra ya esta en el diccionario que se va a devolver como resultado del método, 
	si lo esta se aumenta en uno su valor, si no, se añade la palabra al diccionario con valor 1. De esta 
	manera ya se obtiene el diccionario con la frecuencia de cada palabra de la query . 
    \item TF:	Es el método que calcula el TF de las palabras en la query , en él se llama al método CogerPalabras y al 
	método Repeticiones . El primero para obtener la lista de las palabras de la query y saber el total 
	de palabras de la misma a través del método Count() aplicado sobre la lista y el segundo para 
	tener las frecuencias de las mismas en la query . Acto seguido se itera sobre el diccionario que 
	contiene las frecuencias y se divide el valor de la palabra en el diccionario entre la cantidad total 
	de palabras del texto , el resultado de esta operación se añade al diccionario tfs que es el que va a 
	ser devuelto eventualmente por el método TF . 	
    \item CogerUna : Devuelve un  HashSet que es el resultado de iterar sobre la lista que contiene todas las palabras de la query , 
    obteniendo así el vocabulario de la misma .	  
    \item CopiarHash : Devuelve un diccionario cuya clave es cada 	palabra del vocabulario y su valor 0 . La utilidad de este ultimo radica en que se va a usar para aumentar su frecuencias y guardar los datos del siguiente método. 
	\item NumeroenDocumentos; es el método que calcula en cuantos documentos de la 
	colección aparece cada palabra de la query . Para esto se llama en primer lugar a tres métodos de 
	la clase . CogerPalabras , CogerUna y CopiarHash. Del primero se obtiene la lista con todas las 
	palabras de la query , el segundo recibe como parámetro esta lista y construye el vocabulario de la 
	query y lo guarda en un HashSet y el tercero recibe como parámetro a este HashSet y construye el 
	diccionario(la clave son las palabras y el valor es 0 ) que será devuelto al final del método . Se hace 
	uso del método GetFiles de la clase Directory para almacenar en un array todas las rutas de los 
	documentos de la carpeta Content . A continuación, se itera sobre el Vocabulario de la query ( 
	HashSet del método CogerUna ) y en un segundo ciclo se itera sobre el array con las rutas de los 
	documentos y haciendo uso del método ReadAllText de la clase File se almacena en un string cada 
	texto . Luego se aplica el método Contains a este string y se devuelve true se aumenta en una el 
	valor de la palabra en el diccionario resultante del método CopiarHash . Asi ya se obtiene en un 
	diccionario estos valores que se usaran para el calculo del IDF . 
	\item  IDF : Se utiliza el diccionario que obtenemos del método anterior ( 	llamando al método ) .Luego al igual que antes hacemos uso del método GetFiles de la clase Directory para guardar en un array todas las rutas 	de los archivos ( la utilidad de esto acá es aplicarle el método Length al array obtener la cantidad de documentos en la colección de documentos ) . Ya luego se itera sobre el diccionario frecuencias que es el que almacena los datos del método NumeroenDocumentos ( cantidad de documentos en 	que aparece cada palabra de la query ) . Si el valor de alguna de las palabras es cero se añade con	valor 0 al diccionario que devolverá este método . Si el valor de algunas es igual a la cantidad total 	de documentos de la colección se agrega al diccionario con valor 0.00001 , muy cercano a cero . 	Esto lo hago para evitar que el método SimilitudentreCosenos devuelva NaN en el caso particular de que la query aparezca en todos los documentos . Y ya por último si nada de esto ocurre entonces se calcula el IDF de la palabra . Previamente fuera del bucle se inicializó una variable con el valor del logaritmo natural de la cantidad de documentos de la colección para evitar hacerlo dentro del bucle ya que es un valor constante . Aunque la formula del IDF es Logaritmo Natural de 	la cantidad total de documentos de la colección entre la cantidad de documentos en los que 	aparece la palabra acá hago uso de la propiedad de los logaritmos que afirma que el logaritmo de una división es igual a la resta de los logaritmos de ambos componentes de la división . 	Ya por último el método CalcularTDIDF realiza la operación de multiplicar cada valor de los 	diccionarios que almacenan los valores del TF y del IDF . Se inicializan dos diccionarios llamando a 	los respectivos métodos ya explicados previamente y en un ciclo se multiplican ambos valores . 
	
\end{enumerate}

\subsection{Clase \texttt{Puntuación}}\label{sub:puntuacion}

Esta clase es donde se realizan todos los procesos para llegar al score .

\subsubsection{Métodos}

\begin{enumerate}
	\item  Nombres : Se hace uso del método GetFiles de la clase Directory para 
	almacenar en un array las rutas de los archivos de la carpeta Content . Luego se itera sobre este 
	array y haciendo uso del método GetFileName(como parámetro se la pasa cada ruta del archivo ) 
	se obtiene el nombre de cada archivo , cada nombre se guarda en una lista . Así se han obtenido
	todos los nombres de los documentos de la colección . 
	\item NewCollection : Es el método encargado de crear un vocabulario en común entre la 
	query y el documento , esto en pos de que los dos tengan la misma longitud y poder aplicar sobre 
	ellos la formula de la similitud entre cosenos . Este método recibe dos parámetros que vendrían 
	siendo los correspondientes a los valores del TF-IDF de la query y los documentos y devuelve un 
	diccionario con clave string y una 	tupla de valores , el primer valor de a tupla corresponde a un diccionario y el segundo a otro . 
	Primero se recorre uno de los diccionarios y se añade cada palabra al diccionario que se devolverá 
	eventualmente , además se añade una tupla , el primer valor es el correspondiente a la clave del 
	diccionario que se recorre y el segundo valor es 0.Luego de terminado este proceso en otro ciclo 
	se recorre el segundo diccionario , si el diccionario que va a ser devuelto ya contiene alguna de las 
	claves entonces simplemente se añade el valor a la tupla . Si no contiene a la clave entonces el 
	primer valor de la tupla se agrega en 0 y en el segundo se agrega el valor correspondiente a la 
	clave del diccionario que se esta recorriendo . De esta manera al finalizar se obtiene un diccionario 
	cuyas claves son todas las palabras del vocabulario entre la query y el documento y cuya tupla de 
	valores representan los valores de las claves en los dos diccionarios que se recibieron como 
	parámetro . 
	\item ProductoEscalar : Recibe como parámetro este diccionario anterior y recorriéndolo 
	realiza el producto de ambos valores de la tupla , cada resultado del producto se va agregando 
	dentro del ciclo a una variable que va conteniendo la suma de estos productos , así al final se 
	obtendrá el valor del Producto escalar entre los vectores de la query y el documento . 
	\item ProductoMagnitudVectores : Recibe el mismo diccionario que el anterior método y 
	calcula la magnitud de cada vector . Se recorre el diccionario en un ciclo y en dos variables suma1 
	y suma2 se van almacenando las sumas de los resultados de la operación de elevar al cuadrado 
	cada componente del vector . Aquí cabe aclarar que cada vector acá es un valor de la Tupla . Por 
	ejemplo, se tiene una clave y una tupla , el primer valor corresponde a un vector y el segundo a 
	otro . Luego cuando termina el ciclo se le halla el valor de la raíz cuadrada de cada suma y se 
	multiplican teniendo así el producto de la magnitud de ambos vectores necesario para calcular la 
	similitud entre cosenos . 
	\item SimilitudEntreCosenos : Como parámetro se reciben dos diccionarios , uno 
	correspondiente a los valores del TF-IDF de un documento y el otro correspondiente a la query . Se 
	van a emplear los tres métodos explicados anteriormente : 
	\begin{enumerate}
		\item NewCollection – para construir el vocabulario común entre los dos vectores , igualar su longitud 
		y además almacenarlos en un mismo diccionario . 
		\item ProductoEscalar- Se le pasa como parámetro el diccionario anterior para obtener el producto 
		escalar de ambos vectores 
		\item 	ProductoMagnitudVectores:Tambien se le pasa como parámetro el diccionario de NewCollection 
		y realiza el producto de la magnitud de ambos vectores. 
		Los valores del producto escalar y el producto entre la magnitud de los vectores se almacenan en 
		una variable respectivamente y luego la variable similitud almacena el cociente entre las dos 
		variables anteriores , resultando así la Similitud entre cosenos . 
	\end{enumerate}
 
 	\item  Scores : devuelve una lista de doubles con los valores de la similitud entre cosenos entre 
	la query y todos los documentos de la colección . Primero se realiza una copia del objeto tipo 
	matriz, así como de la query para evitar errores a la hora de recorrerlos en un ciclo . En un bucle 
	foreach se recorre cada diccionario en la copia de la matriz y se calcula la similitud entre cosenos 
	entre el diccionario ( el vector de cada documento ) y el vector de la query , se va añadiendo cada 
	valor a la lista con los scores. 
    \item Union : Con este método se une las listas resultantes de los métodos Nombres y Scores , de manera 
	que se obtenga un diccionario en el que a cada archivo ( su nombre) se le asocie un score. 
	\item  OrdenarScores : Se le aplica a un objeto tipo Puntuación con el objetivo de ordenar los 
	scores de mayor a menor . Primero se crea una lista y se recorre el objeto tipo Puntuación de 
	manera de que a la lista se le vaya añadiendo los valores de los scores que se encuentran en el 
	diccionario . Cuando termina el ciclo se le aplica el método Sort a la lista quedando así ordenados 
	de menor a mayor . Luego en un ciclo se va a recorrer la lista empezando por el ultimo valor ( el 
	mayor ) y terminando por el primero ( el menor ) . Dentro de este mismo ciclo se va a recorrer el 
	objeto tipo Puntuación . Si un valor de la lista coincide con un valor del diccionario ( objeto tipo
	Puntuación ) entonces si el diccionario que se va a devolver ( llamémoslo orden ) , si orden ya 
	contiene esa clave no se va a hacer nada , si no la contiene entonces se añade la clave con su valor 
	de score correspondiente . De esta manera se obtiene un diccionario donde a cada titulo se le 
	asocio su valor del score , y además en orden . 
	\item  Snippet : Devuelve un diccionario en el que a dos claves ( nombre y snippet ) se le 
	asocia un valor (score). 
	Se le pasa un diccionario como parámetro que será el que contenga los scores ordenados . Luego 
	se itera sobre este diccionario . Aquí se inicializa una variable tipo string (llamémosla texto) , se 
	utiliza el método string.Format para añadirle a la ruta de la carpeta Content el nombre de cada 
	archivo de manera tal que a la variable que se inicializa luego 
	usando el método ReadAllText de la clase File ( se le pasa como parámetro la ruta del documento ) .
    Luego hacemos texto.IndexOf(se le pasa como parámetro un 
	punto indicando que devuelva la posición del primer punto y además se vuelve a llamar a 
	texto.Index (se le pasa como parámetro un punto ) – esta vez indicando que devuelva la posición 
	del primer punto después del primer punto encontrado ), de esta manera se obtiene la posición 
	del segundo punto en el texto , delimitando así las dos primeras oraciones. Luego se inicializa otra 
	variable string y se llama al método substring ( texto.Substring ( 0 , posición del segundo punto +1 
	) , indicando que archive en la variable las dos primeras oraciones del texto . Ya por ultimo se 
	añade esta subcadena al diccionario como la segunda clave . Quedando estructurado de la 
	siguiente manera : Nombre , snippet : score. 
	\item  Método constructor : Se le pasa como parámetro un objeto tipo matriz y un objeto 
	tipo Query . En el interior se inicializan en primer lugar dos variables : una lista de string resultado 
	de llamar al método Nombres ( con todos los nombre de los archivos ) y otra lista de dobles ( con 
	todos los scores resultado de llamar al método Scores).Luego se inicializa un Diccionario de string y 
	double con el método Union de manera que cada nombre de archivo quede asociado a su score . 
	
    \item  Query :Justo afuera del método Query en la clase Moogle se inicializa la variable estática matriz que es un 
	objeto tipo Matriz ( Lista de Diccionarios con los valores de los TF-IDF de cada palabra de cada 
	documento de la colección ) .Cuando se llama al Método Query se inicializa un objeto tipo Query , en donde se procesa la query 
	como se ha explicado previamente . Luego se inicializa un objeto tipo Puntuación (llamémosle 
	scores ), este sería el Diccionario con los valores de los scores asignados a cada texto . Luego se 
	inicializa un diccionario con el método scores.OrdenarScores(); aquí se obtienen los valores de los 
	scores ordenados de mayor a menor . Después de esto se llama el método estático de la clase 
	Puntuación al que se le pasa como parámetro el diccionario anterior , de acá se obtiene un 
	diccionario con el snippet integrado , dos claves ( nombre,snippet) y el valor del score . 
	Ya tenemos lo que seria el resultado de la búsqueda , nos falta integrarlo , por decirlo de alguna 
	manera al array items de objetos tipo SearchItem . Para esto inicializamos uno ( con el Length igual 
	al Count del diccionario con los snippet integrados ) . Luego en un ciclo recorremos el diccionario y 
	con un contador vamos agregando los tres elementos al array ( los parámetros que se le pasa al 
	constructor de SearchItem ). 
	Finalmente se devuelve un objeto tipo SearchResult cuyo constructor recibe dos parámetros , el 
	array items y la query .
	
\end{enumerate}

\newpage
\section{Conclusiones}\label{sec:conclusiones}
Definitivamente el proyecto ha sido una experiencia muy valiosa para el aprendizaje y la práctica de la programación. El manejo de estructuras de datos como los arrays, las listas y los diccionarios resulta fundamental en la mayoría de los proyectos de programación, y el hecho de haber trabajado con ellos en este proyecto permitió mejorar las  habilidades en la manipulación de datos.

Asimismo, el proyecto representó una oportunidad única para aprender sobre técnicas de procesamiento de lenguaje natural, en particular el TF-IDF, que es una técnica muy importante en la búsqueda y recuperación de información. El haber desarrollado un navegador que utiliza esta técnica   permite aplicar estos conocimientos en futuros proyectos relacionados con la búsqueda de información.

Por otro lado, el haber enfrentado por primera vez la realización de un proyecto complejo fue sin duda una experiencia enriquecedora y exigente. Brindó la oportunida de  aplicar conocimientos, resolver problemas y aprender de los errores, lo que permitirá mejorar las habilidades en el desarrollo de proyectos más complejos en el futuro.

En conclusión, el proyecto representó una oportunidad única para el aprendizaje y la práctica de la programación, y permitió  mejorar significativamente las habilidades en el manejo de estructuras de datos y en el procesamiento de lenguaje natural.








\end{document}
   
   
	 
	